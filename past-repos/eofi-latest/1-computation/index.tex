% .9 done on index contents
\chapter{Computation}

The only thing that's reliable is the definition. The only linkage or the connection is the reuse of the definition.

There are only two types of information: data and function. while we can consider function as a piece of data. and the third one is introduced as abstraction  

% Introduction
% \subsection{Verificable Definition}
\subsection{Ability and Feature}
\subsection{Relationships}
\subsection*{Real-Life Examples}
Let's say you would like to use command line 'git clone https://github.com/lin/eofi.git', while this is exactly distinguishability is. The whole idea of computer science is based on distinguishability.

Any identifier is used to distinguish from each other. Your username in the internet is to distinguish you from others. The number system is to distinguish things, like you ID number. We use number to distinguish things. Address numbers. IP address. The binary system can be used to distinguish things. (00, 01, 10, 11)

This is mapping distinguishable things to another distinguishable things. This is also the case of replaceable definition.

The following is for ambiguity, not for distinguishability.

Another great example is the capital letters. GoT or got,

\begin{quote}
- Are you real? \\
- Well, if you can't tell, does it matter?
\end{quote}



% Definition
% \section{Replaceable Definition}
\subsection{Reference and Dereference}

% Composition
\section{Composition}
% \subsection{Array and String}
% \subsection{Hash Object}

% Transformation
\section{Transformation}

One thing I have to remember is that you have to make the instruction work for all instant. This is the foundation of this book. You have to make abstraction before you make transformation.

The transformation you define is for all not for instance. Even in turing machine, you move left twice, you don't care about the one you missed, because that is not what you care, no matter what symbol is on that square, it's the same instruction. This is the abstraction of operation, on operation to deal with all situations.

What intuitive really means is the automatic autocorrection based on the event trasnformational rules in brains, that processed in the subconscious. And the stregth of the neurons is how intuitive it will be.
\subsection{Events}

The first three are from an article by Chaitin\cite{chaitin}.

% \begin{figure}
%   \includegraphics[width=\linewidth]{img/physics.png}
%   \caption{Computation Models}
%   \label{fig:phyics}
% \end{figure}
%
% \begin{figure}
%   \includegraphics[width=\linewidth]{img/math.png}
%   \caption{Computation Models}
%   \label{fig:math}
% \end{figure}
%
% \begin{figure}
%   \includegraphics[width=\linewidth]{img/computer.png}
%   \caption{Computation Models}
%   \label{fig:computer}
% \end{figure}

% \[
%   \text{Theory}\xrightarrow{\displaystyle \text{Calculations}}\text{Predictions for Observations}
% \]
%
% \[
%   \text{Axioms}\xrightarrow{\displaystyle \text{Reasoning}}\text{Theorems}
% \]
%
% \[
%   \text{Program}\xrightarrow{\displaystyle \text{Execution on Computer}}\text{Output}
% \]
%
%
% \[
%   \text{Premise}\xrightarrow{\displaystyle \text{Rules}}\text{Conclusions}
% \]

\begin{figure}
  \includegraphics[width=\linewidth]{img/base.png}
  \caption{Computation Models}
  \label{fig:base}
\end{figure}

The rules mean under what conditions do what.

By saying conditions, first you have to know how to recognize conditions and what is recognizable conditions, of course that's when you definable, which in turn means distinguishable.

By saying doing, that means you have to know what is allow to transform.

\begin{figure}
  \includegraphics[width=\linewidth]{img/computing.png}
  \caption{Schematic Definition of Computation}
  \label{fig:computation}
\end{figure}

If we have clearly defined the \textbf{computing} procedure, we can easily compute. That means the input is distinguishable from others and the transformation rule is clear and distinguishable.
For example:
\begin{verbatim}
  def add(x)
    return x + 1
\end{verbatim}

And your goal is to compute:

\begin{verbatim}
  add(3)
\end{verbatim}

\begin{figure}
  \includegraphics[width=\linewidth]{img/learning.png}
  \caption{Schematic Definition of Learning}
  \label{fig:learning}
\end{figure}

Whereas \textbf{learning} is that you know the input and output, but you don't know the transformation rules.

For example:
\begin{verbatim}
  add(3) = 4
  add(4) = 5
  add(5) = 6
  add(6) = 7
\end{verbatim}

And your goal is to compute:

\begin{verbatim}
  add(10)
\end{verbatim}

\begin{figure}
  \includegraphics[width=\linewidth]{img/solving.png}
  \caption{Schematic Definition of Solving}
  \label{fig:solving}
\end{figure}

\textbf{Searching} is to find an instance which satisfies certain criteria in terms of computing algorithm.

For example:
\begin{verbatim}
  def add(x)
    return x + 1
  add(m) = 5
\end{verbatim}

And your goal is to compute:

\begin{verbatim}
  m
\end{verbatim}

\subsection{Inputs and Outputs}
% \subsection{Benefits of Event-Based View}

% Computation
% \subsection{Verificable Definition}
\subsection{Ability and Feature}
\subsection{Relationships}
\subsection*{Real-Life Examples}
Let's say you would like to use command line 'git clone https://github.com/lin/eofi.git', while this is exactly distinguishability is. The whole idea of computer science is based on distinguishability.

Any identifier is used to distinguish from each other. Your username in the internet is to distinguish you from others. The number system is to distinguish things, like you ID number. We use number to distinguish things. Address numbers. IP address. The binary system can be used to distinguish things. (00, 01, 10, 11)

This is mapping distinguishable things to another distinguishable things. This is also the case of replaceable definition.

The following is for ambiguity, not for distinguishability.

Another great example is the capital letters. GoT or got,

\begin{quote}
- Are you real? \\
- Well, if you can't tell, does it matter?
\end{quote}



% CTT
\subsection{Boundary of Computation}
\subsection*{Turing Completeness}

What I guess is that the reason of the robustness of Turing machine is that it can simulation any well-defined thought, the distinguishability at its highest level. No ambiguity is allowed. Everything you can distinguish clearly can be represented in the system. Every rule you can say precisely can be emulated in the machine. Every transformation  every possible computation is within the power of a Turing machine.
% \subsection{Turing Machine and $\lambda$-Calculus}
\subsubsection*{Church-Turing Thesis}

The Turing machine are capable of doing following abilities:

1. store and retrive data.
2. transform  data to another one.
3. able to compute in a sequencial order.
4. able to jump to a specific instruction position.

% \subsection{Verificable Definition}
\subsection{Ability and Feature}
\subsection{Relationships}
\subsection*{Real-Life Examples}
Let's say you would like to use command line 'git clone https://github.com/lin/eofi.git', while this is exactly distinguishability is. The whole idea of computer science is based on distinguishability.

Any identifier is used to distinguish from each other. Your username in the internet is to distinguish you from others. The number system is to distinguish things, like you ID number. We use number to distinguish things. Address numbers. IP address. The binary system can be used to distinguish things. (00, 01, 10, 11)

This is mapping distinguishable things to another distinguishable things. This is also the case of replaceable definition.

The following is for ambiguity, not for distinguishability.

Another great example is the capital letters. GoT or got,

\begin{quote}
- Are you real? \\
- Well, if you can't tell, does it matter?
\end{quote}


