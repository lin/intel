\section{Computation}
\subsection{Read, Write and Store Information}
\subsection{Conditional Checking and Jump}
Here is quite important. If you are using the conditional check, you require more about the data, more specific part of the information, more abilities is on the requirements. So, in this case, you are define a new class of things, you implicitly that this transformation will do different(distinguishability) for different class, so be aware. But if there is no conditional check, that means you don't care about the data content, only its distinguishability. So any example will do the same.

The transformation doesn't care, doesn't bother.

So the transformation is care is the distinguishability in the context(code block).

When you follow the steps of the instruction, you can sense it (of course this can be don mechanically) some things remain and some things don't change the picture.
\subsection*{Real-Life Examples}
Let's say you would like to use command line 'git clone https://github.com/lin/eofi.git', while this is exactly distinguishability is. The whole idea of computer science is based on distinguishability.

Any identifier is used to distinguish from each other. Your username in the internet is to distinguish you from others. The number system is to distinguish things, like you ID number. We use number to distinguish things. Address numbers. IP address. The binary system can be used to distinguish things. (00, 01, 10, 11)

This is mapping distinguishable things to another distinguishable things. This is also the case of replaceable definition.

The following is for ambiguity, not for distinguishability.

Another great example is the capital letters. GoT or got,

\begin{quote}
- Are you real? \\
- Well, if you can't tell, does it matter?
\end{quote}

