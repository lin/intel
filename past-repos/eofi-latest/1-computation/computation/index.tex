\section{Computation}
\subsection{Read, Write and Store Information}
\subsection{Conditional Checking and Jump}
Here is quite important. If you are using the conditional check, you require more about the data, more specific part of the information, more abilities is on the requirements. So, in this case, you are define a new class of things, you implicitly that this transformation will do different(distinguishability) for different class, so be aware. But if there is no conditional check, that means you don't care about the data content, only its distinguishability. So any example will do the same.

The transformation doesn't care, doesn't bother.

So the transformation is care is the distinguishability in the context(code block).

When you follow the steps of the instruction, you can sense it (of course this can be don mechanically) some things remain and some things don't change the picture.
\subsection*{Examples}
Here lists some examples of mathematical classifications:

\begin{example}
  Give an instance of prime numbers
\end{example}

\begin{example}
  Is 3 a prime number?
\end{example}

\begin{example}
  Is $f(x) = x^3$ an odd function?
\end{example}

\begin{example}
  What is the fourth elements of $a_n = n^2$
\end{example}

Odd number is the best illustration of abstraction.

