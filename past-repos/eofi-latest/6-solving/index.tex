\chapter{Solving}

% \subsection{Verificable Definition}
\subsection{Ability and Feature}
\subsection{Relationships}
\subsection*{Real-Life Examples}
Let's say you would like to use command line 'git clone https://github.com/lin/eofi.git', while this is exactly distinguishability is. The whole idea of computer science is based on distinguishability.

Any identifier is used to distinguish from each other. Your username in the internet is to distinguish you from others. The number system is to distinguish things, like you ID number. We use number to distinguish things. Address numbers. IP address. The binary system can be used to distinguish things. (00, 01, 10, 11)

This is mapping distinguishable things to another distinguishable things. This is also the case of replaceable definition.

The following is for ambiguity, not for distinguishability.

Another great example is the capital letters. GoT or got,

\begin{quote}
- Are you real? \\
- Well, if you can't tell, does it matter?
\end{quote}



% \subsection{Verificable Definition}
\subsection{Ability and Feature}
\subsection{Relationships}
\subsection*{Real-Life Examples}
Let's say you would like to use command line 'git clone https://github.com/lin/eofi.git', while this is exactly distinguishability is. The whole idea of computer science is based on distinguishability.

Any identifier is used to distinguish from each other. Your username in the internet is to distinguish you from others. The number system is to distinguish things, like you ID number. We use number to distinguish things. Address numbers. IP address. The binary system can be used to distinguish things. (00, 01, 10, 11)

This is mapping distinguishable things to another distinguishable things. This is also the case of replaceable definition.

The following is for ambiguity, not for distinguishability.

Another great example is the capital letters. GoT or got,

\begin{quote}
- Are you real? \\
- Well, if you can't tell, does it matter?
\end{quote}


How intimidating it is for a newcomer to a new city.

\section{Principle of Heuristics}
\subsection{Computational Cost Saving}

\section{The Analogy of City Exploration}

Since to solve is to search, searching is basically exploring your searchable area.

You start with a newly assigned treasure hunt game. You start you search by going through the known routes, following the familiar signposts, try the nearby stores to save your energy, you find a famous gathering place like a museuem or a park.

You start to find the easter egg by visiting the hinted places, and you get there by some known and familiar routes, you can search nearby the hinted places, since you know how to get there and assume you can find there will be considered as being lucky. Sometimes you have to find your own routes by noticing the signposts. And since you might visit the city multiple times, you will gradually remember the notable objects. When you on a road, you will automatically complete the routes without thinking or even chatting with others, until a puzzled crossroad is encountered.

Let's compare the solving route as how much roads you have to travel. And the fluency of your technique mastering is a value of resistance, low value with well trained, and high for poor practice. That might be the case in our brain that the smooth of a chunk is basically the electric current with higher value. And the brain choose load through those low resistance route to save computation energy.

\begin{figure}
  \centerline{\includegraphics[width=1.2\linewidth]{img/range.pdf}}
  \caption{Types of Transformation}
  \label{fig:range}
\end{figure}


% \subsection{Verificable Definition}
\subsection{Ability and Feature}
\subsection{Relationships}
\subsection*{Real-Life Examples}
Let's say you would like to use command line 'git clone https://github.com/lin/eofi.git', while this is exactly distinguishability is. The whole idea of computer science is based on distinguishability.

Any identifier is used to distinguish from each other. Your username in the internet is to distinguish you from others. The number system is to distinguish things, like you ID number. We use number to distinguish things. Address numbers. IP address. The binary system can be used to distinguish things. (00, 01, 10, 11)

This is mapping distinguishable things to another distinguishable things. This is also the case of replaceable definition.

The following is for ambiguity, not for distinguishability.

Another great example is the capital letters. GoT or got,

\begin{quote}
- Are you real? \\
- Well, if you can't tell, does it matter?
\end{quote}


% \section{Possible Hidden Places}
\subsection{Solve Like a Fox}
Don't try to generalize too early, and don't try to be perfect early on. The reason is simple, the cost is too much and the resource is limited. We can explore everything and make perfect choice at each step, the only way we could do is moving along. The wisdom is not even doing your best but to do the thing you can do, making progress instead of waiting for the right time or making the perfect match to your problem.

You can come up an idea and try it, then modify it later, and even you are improving it by visiting it multiple times. This strategy is simple but maybe powerful. It released the tension of limited resources. And it guesses along or maybe even believe its own ability that the result is at the reach and improve its understanding (I should define this word more precisely) along the way.

Solving a problem with ad hoc techniques may give the first hint of generalizations. But this route might be easier using the brutal force method rather the abst ract way thinking. We'd better revist the problem after solving, which will be a separate topic in the future.

The way of heuristics works in this book is that: in the indistinguishable path choosing situation, by assuming something may work,

\subsection{Increase the Similiarity}

\begin{example}
  If $2a+2^a = 5, 2b+ 2\log_2{(b-1)} = 5$, what is $a+b$?
\end{example}

You can solve $a$, which is around $1.28315$, and you can also get $b$, approximately $2.21685$. And the addition of these two numbers is exactly $3.5$. The important part of the transformation is to change the first part to $2a + 2 \cdot 2^{a-1} = 5$, so the two expression can have a symmetric look.

% \subsection{Verificable Definition}
\subsection{Ability and Feature}
\subsection{Relationships}
\subsection*{Real-Life Examples}
Let's say you would like to use command line 'git clone https://github.com/lin/eofi.git', while this is exactly distinguishability is. The whole idea of computer science is based on distinguishability.

Any identifier is used to distinguish from each other. Your username in the internet is to distinguish you from others. The number system is to distinguish things, like you ID number. We use number to distinguish things. Address numbers. IP address. The binary system can be used to distinguish things. (00, 01, 10, 11)

This is mapping distinguishable things to another distinguishable things. This is also the case of replaceable definition.

The following is for ambiguity, not for distinguishability.

Another great example is the capital letters. GoT or got,

\begin{quote}
- Are you real? \\
- Well, if you can't tell, does it matter?
\end{quote}



\section{Variable-Based Strategies}
\subsection{Completeness of Information}

\section{Trial-Based Induction}

\section{Multi-Layer Exploration}

\section{The Generic Solving Algorithm}
\subsection{Frequecy of Usage}
\subsection{Comfort of Familiarness}
\subsection{Toolbox Hierarchy}
Analogy of a memory hierarchy.
\subsection{Automatic Exploration}

\section{Reviewing the Solution}
\subsection{Hardness of a Problem}
\subsection{Evaluation of the Solution}
\subsection{Improvements of the Solution}
\subsection{Improvements of the Solving Ability}

% Real World Problems
% Not sure should I include these.
% \subsection{Verificable Definition}
\subsection{Ability and Feature}
\subsection{Relationships}
\subsection*{Real-Life Examples}
Let's say you would like to use command line 'git clone https://github.com/lin/eofi.git', while this is exactly distinguishability is. The whole idea of computer science is based on distinguishability.

Any identifier is used to distinguish from each other. Your username in the internet is to distinguish you from others. The number system is to distinguish things, like you ID number. We use number to distinguish things. Address numbers. IP address. The binary system can be used to distinguish things. (00, 01, 10, 11)

This is mapping distinguishable things to another distinguishable things. This is also the case of replaceable definition.

The following is for ambiguity, not for distinguishability.

Another great example is the capital letters. GoT or got,

\begin{quote}
- Are you real? \\
- Well, if you can't tell, does it matter?
\end{quote}



% \section{Real-Life Solving}
% \subsection{Hierarchy of Human Needs}
% \subsection{Complexity of Human Aesthetic}
% \subsection{Sensibility of Human Aesthetic}
% \subsection{Alternative Reality}

\section{Designing the Solver}
\subsection{Benefits of the Solver}
\subsection{Revolution on the Foundation of Mathematics}
\subsection{Universal Algorithm of Distinguishability}
\subsection{Reinvent the Wheel}
\subsection{The End of Road}

The only problem we need to solve to solve all problems is the problem of problem solving.

So, why not solve it?
