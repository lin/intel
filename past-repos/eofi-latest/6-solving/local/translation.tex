\subsection{Translation}

In lots of situation, the proof of one result requires long sequences of derivations.
\[
  u \Rightarrow u_1 \Rightarrow u_2 \Rightarrow \dots \Rightarrow u_k \Rightarrow v
\]

But because of the frequency of usage, we tend to remember it as a shortcut.

\[ u \overset{*}{\Rightarrow} v\]

We can call these derivations as translation, since it seems exotic and requires recognization and transform to another string. And for translation, we could name the list of translations as vocabulary. To distinguish them by the frequency of usage, we could name them as, \textbf{basic vocabulary}, \textbf{essential vocabulary}, \textbf{advanced vocabulary}, \textbf{extended vocabulary} etc.

Here lists some examples of mathematical vocabulary: 

\begin{example}
  In $\Delta ABC$, $D$ is the middle point of $BC$, and $BC=2AD \Rightarrow \angle A = 90^\circ$
\end{example}

\begin{example}
  In $\Delta ABC$, point $M$ satifies $\overrightarrow{MA} + \overrightarrow{MB} + \overrightarrow{MC} =\mathbf{0}$  $\Rightarrow M$  is the geometric center of $\Delta ABC$.
\end{example}

\begin{example}
  In $\Delta ABC$, $AB=AC \Rightarrow \angle B = \angle C$
\end{example}

\begin{example}
  Point $P (x_0, y_0)$ is on curve $y = f(x)$ $\Rightarrow y_0 = f(x_0)$
\end{example}

\begin{example}
  If points $A, B, C$ is on the same line $\Rightarrow k_{AB} = k_{BC}$
\end{example}

\begin{example}
  $\overrightarrow{PF_1} \cdot \overrightarrow{PF_2} = 0 \Rightarrow P$ is on the circle whose diameter is $F_1 F_2$
\end{example}

\begin{example}
  Points $A, B$ is on $\dfrac{x^2}{a^2} + \dfrac{y^2}{b^2} = 1$ and $M$ is the middle point of $AB \Rightarrow k_{AB} \cdot k_{OM} = -\dfrac{b^2}{a^2}$
\end{example}

\begin{example}
  $\sin{\dfrac{\pi}{12}} \Rightarrow \dfrac{\sqrt{6} -\sqrt{2}}{4}$
\end{example}


%%%%%%%%%%%%%%%%%%%%%%%%%%%
%%%%%%%%%%%%%%%%%%%%%%%%%%%
%%%%%%%%%%%%%%%%%%%%%%%%%%%
%%%%%%%%%%%%%%%%%%%%%%%%%%%


\subsection{Decoding}

Sometimes, we don't always know the association between two propositions. We need to find it out by ourselves.

\begin{example}
  Given $f(x) = |\log{x}|$ and $f(a) = f(b)$
\end{example}

% \begin{enumerate}
%   \item
%   \item
%   \item
%   \item
% \end{enumerate}
