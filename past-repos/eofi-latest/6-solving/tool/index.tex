\section{Tool-Based Adaption}
\subsection{Creation of Desired Vision}
\subsection*{An Example}
Let's start with an example:
\begin{example}
  If $x + 2y =3$, what's the minimum of $\dfrac{2}{x} + \dfrac{1}{y}$ ?
\end{example}

Let's make us understand the problem. The condition says that $x, y$ satisfies:

\begin{verbatim}
def some_class(x, y)
  return x + y == 3
\end{verbatim}

Then, what is the minimum value comes out from:

\begin{verbatim}
def some_computation(x, y)
  return 2 / x + 1 / y
\end{verbatim}

Let's combine these two:

\begin{verbatim}
def find_some_value()
  all_x, all_y = all_real_numbers()
  all_satisfied_instances = []
  for x in all_x and for y in all_y
    if (x + y == 3)
      all_satisfied_instances.append( 2 / x + 1 / y )
  return find_maximum_value_of(all_satisfied_instances)
\end{verbatim}

This is insolvable numerically, it can only be solved symbolically in terms of class level of manipulation.

Here we go. First the \textit{find\_some\_value} function is equivalent to:
\begin{verbatim}
def find_some_value()
  all_x, all_y = all_real_numbers()
  all_satisfied_instances = []
  for x in all_x and for y in all_y
    # ======== CHANGED ======== #
    if (1 / 3 * (x + y) == 1)
    # ======== CHANGED ======== #
      all_satisfied_instances.append( 2 / x + 1 / y )
  return find_maximum_value_of(all_satisfied_instances)
\end{verbatim}

Then,

\begin{verbatim}
def find_some_value()
  all_x, all_y = all_real_numbers()
  all_satisfied_instances = []
  for x in all_x and for y in all_y
    if (1 / 3 * (x + y) == 1)
      # ======== CHANGED ======== #
      all_satisfied_instances.append( ( 2 / x + 1 / y ) * 1 )
      # ======== CHANGED ======== #
  return find_maximum_value_of(all_satisfied_instances)
\end{verbatim}

Then,

\begin{verbatim}
def find_some_value()
  all_x, all_y = all_real_numbers()
  all_satisfied_instances = []
  for x in all_x and for y in all_y
    if (1 / 3 * (x + y) == 1)
      # ======== CHANGED ======== #
      all_satisfied_instances.append( ( 2 / x + 1 / y ) * 1 / 3 * (x + y) )
      # ======== CHANGED ======== #
  return find_maximum_value_of(all_satisfied_instances)
\end{verbatim}

Then,

\begin{verbatim}
def find_some_value()
  all_x, all_y = all_real_numbers()
  all_satisfied_instances = []
  for x in all_x and for y in all_y
    if (1 / 3 * (x + y) == 1)
      # ======== CHANGED ======== #
      all_satisfied_instances.append( 1 + 1 / 3 * ( 2 * y / x + x / y ))
      # ======== CHANGED ======== #
  return find_maximum_value_of(all_satisfied_instances)
\end{verbatim}

Then,

\begin{verbatim}
def find_some_value()
  all_x, all_y = all_real_numbers()
  all_satisfied_instances = []
  for x in all_x and for y in all_y
    if (1 / 3 * (x + y) == 1)
      # ======== CHANGED ======== #
      all_satisfied_instances.append( 2 * y / x + x / y )
  return 1 + 1 / 3 * find_maximum_value_of(all_satisfied_instances)
  # ======== CHANGED ======== #
\end{verbatim}

Then,

\begin{verbatim}
def find_some_value()
  all_x, all_y = all_real_numbers()
  all_satisfied_instances = []
  for x in all_x and for y in all_y
    # ======== CHANGED ======== #
    # ======== CHANGED ======== #
    all_satisfied_instances.append( 2 * y / x + x / y )
  return 1 + 1 / 3 * find_maximum_value_of(all_satisfied_instances)
\end{verbatim}

Then,

\begin{verbatim}
  def find_some_value()
    maimum_of_known_techniques = find_maximum_value_by_x_y('2 * y / x + x / y')
    return 1 + 1 / 3 * maimum_of_known_techniques
\end{verbatim}

You may find it confusing, but this is neccessary if you think deep and try to make it accurate.

As you can see in this example, it's quite complicated to speak in terms of computing. And human brains would have compute this subconstiously or vaguely. If you would like to proof each equivalency, it will take lots of time and that's not feasible in computing. But maybe that's why human brains are so awesome.

And the reason that this won't work is we have to use more distinguishable way to reduce the calculation or in terms of function we need.

This transformation of data is quite unpleasing and troublesome. It simply so hard to even follow and to reasoning. While human might deduce with error and check it after the trial. We may have imitate that in our A.I. but make this A.I. more

\subsection{Derivatives}

When you want to know the result of $\overrightarrow{AC} - \overrightarrow{BC} $, you know it's either $\overrightarrow{AB}$ or $\overrightarrow{BA}$, but how do you get the truth?

Well, you would remember it through a remembered rule, like first to second, so it's $\overrightarrow{AB}$. But you can also get through

\subsubsection*{Analogy from Startups}
When something phenomenal happens, people will start ask what ideas can also work by knowing the success of such improvements.

The success of Facebook gives people the imagination of how to solve problems by working in the dorm room.

The success of Uber and Airbnb, let people think about other opportunities by online sharing.

\subsection{Declaritive Derivations}
Let's consider two properties of an arithmetic sequence:
\begin{enumerate}
  \item $S_{2n-1} = (2n-1) a_n$, for $n > 1$
  \item $a_m + a_n = a_p + a_q$, if $m +n = p + q$
\end{enumerate}

The condition of the followin problems is:
\[ S_7 = 7 \]

The direct use of the first tool:
\[ \text{Find: } a_4 \]

The combination of two tools:
\[ \text{Find: } a_2 + a_4 + a_6 \]

Here comes a little bit hard problem:
\[ \text{Find: } a_1 + a_5 + a_6 \]

In which case, you have to transform the first two to $a_2 + a_4$, then the problem is reduced to the second one. But how can you think of this? In front of you is unlimited possible paths to choose. Even the reasonable ones will also be a lot. But why should we choose change the first two operands? Why we have to increase the first and decrease the second? It's solveable because of the small amount of the search area, but how can we increase the searching efficiency?

Here is the reason: We want to use the first properties since we know $S_7 = 7$, we get $a_4 = 1$ at hand. So we would like to expose the appearance of $a_4$ to explore more. With this in mind, we can also change the latter two operands to $a_4+ a_7$, through which we can also solve the problem at hand.

Now, let's consider a more difficult one:

\[ \text{Find: } 3a_3 +a_7 \]

We can also introduce $a_4$ by changing it to $a_3+ a_4 + a_2 + a_7$, now the problem reduces to $a_3+a_2+a_7$, which is $a_1 + a_4 + a_7$.

After a few trial, we can induce that if the addition of all subscripts is divisible by 4, then the problem is solvable. The strategy would be keeping introducing the known tool of $a_4$.
