\subsection{Symbolic Computations}

Using the symbols to denote an instance with special properties.

\begin{example}
  If the addition of the imaginary and real part of $z = a + 2ai$ is 6, what is $a$?
\end{example}

\begin{example}
  If $\log_2{x} = 3$, what is $x$?
\end{example}

\begin{example}
  If the left focus of $\dfrac{x^2}{4} + \dfrac{y^2}{b^2} = 1$ is $(1, 0)$, what is $b$?
\end{example}

\begin{example}
  $\displaystyle \int_1^a x^2 \, dx = 9 $, what is $a$?
\end{example}

\begin{example}
  If the average of $a, 2, 3, 1, 6$ is $4$, what is $a$?
\end{example}

Following examples require setting up more equations.

\begin{example}
  For arithmetic sequence $a_n$, if $a_3 = 3$, $S_4=10$, what is $a_n$
\end{example}

\begin{example}
  If a circle passes points $(1,2), (3,4), (5,6)$, what is the equation of the circle?
\end{example}
